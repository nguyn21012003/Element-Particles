\documentclass{article}
\usepackage[utf8]{vietnam}
\usepackage[utf8]{inputenc}
\usepackage{anyfontsize,fontsize}
\changefontsize[13pt]{13pt}
\usepackage{commath}
\usepackage{blindtext}
\usepackage{parskip}
\usepackage{xcolor}
\usepackage{amssymb}
\usepackage{slashed}
\usepackage{indentfirst}
\usepackage{pdfpages}
\usepackage{graphicx}

\usepackage{tikz-feynman}
\usepackage{nccmath}
\usepackage{mathtools}
\usepackage{amsfonts}
\usepackage{amsmath,systeme,bbold}
\usepackage[thinc]{esdiff}
\usepackage{hyperref}
\usepackage{dirtytalk,bm,physics}
\usepackage{tikz}
\usepackage{lipsum}
\usepackage{fancyhdr}
%footnote
\pagestyle{fancy}
\renewcommand{\headrulewidth}{0pt}%
\fancyhf{}%
\fancyfoot[L]{Vật lý Lý thuyết}%
\fancyfoot[C]{\hspace{4cm} \thepage}%

\usetikzlibrary{shapes.geometric, arrows}

\usepackage{geometry}
\geometry{
	a4paper,
	total={170mm,257mm},
	left=20mm,
	top=20mm,
}

\renewcommand{\baselinestretch}{2.0}
\usetikzlibrary{arrows.spaced}
\usetikzlibrary{animations,quotes}
%gian do
\tikzstyle{startstop} = [rectangle, rounded corners, minimum width=3cm, minimum height=1cm, text centered,draw=black, fill=white!30]
\tikzstyle{arrow} = [thick,->,>=stealth]

\title{\Huge{BTVN 2}}

\hypersetup{
	colorlinks=true,
	linkcolor=red,
	filecolor=magenta,      
	urlcolor=cyan,
	pdftitle={},
	pdfpagemode=FullScreen,
}

\urlstyle{same}

\begin{document}
\setlength{\parindent}{20pt}

\author{TRẦN KHÔI NGUYÊN \\ VẬT LÝ LÝ THUYẾT}
\maketitle

Chứng minh $\displaystyle\sum_{s=1}^{2} v_{\vec{k}}^{(s)} \bar{v}_{k}^{(s)} = -\slashed{k} + m $
\begin{align*}
	RHS
	 & = \sqrt{E_{\vec{k}} + m }
	\begin{pmatrix}
		\frac{\vec{k}\vec{\sigma}}{E_{\vec{k}} + m } \chi^{(s)} \\
		\chi^{(s)}
	\end{pmatrix}
	\sqrt{E_{\vec{k}} + m }
	\begin{pmatrix}
		\frac{\vec{k}\vec{\sigma}}{E_{\vec{k}} + m } \chi^{(r)^{\dagger} } & \chi^{(r)^{\dagger} }
	\end{pmatrix}\gamma^0                                                                                                                                                                                                          \\
	 & = 	(E_{\vec{k}} + m)
	\begin{pmatrix}
		\frac{\vec{k}\vec{\sigma}}{E_{\vec{k}} + m } \chi^{(s)} \\
		\chi^{(s)}
	\end{pmatrix}
	\begin{pmatrix}
		\frac{\vec{k}\vec{\sigma}}{E_{\vec{k}} + m } \chi^{(r)^{\dagger} } & -\chi^{(r)^{\dagger} }
	\end{pmatrix}                                                                                                                                                                                                       \\
	 & =(E_{\vec{k}} + m)
	\begin{pmatrix}
		\frac{\vec{k}\vec{\sigma}}{E_{\vec{k}} + m } \chi^{(s)} \frac{\vec{k}\vec{\sigma}}{E_{\vec{k}} + m } \chi^{(r)^{\dagger} } & - \frac{\vec{k}\vec{\sigma}}{E_{\vec{k}} + m } \chi^{(s)}\chi^{(r)^{\dagger} } \\
		\chi^{(s)} \frac{\vec{k}\vec{\sigma}}{E_{\vec{k}} + m } \chi^{(r)^{\dagger} }                                              & -\chi^{(s)}\chi^{(r)^{\dagger}}
	\end{pmatrix} \\
	 & =(E_{\vec{k}} + m)
	\begin{pmatrix}
		\frac{k^2}{(E_{\vec{k}} + m)^2 } \mathbb{1}_{2\times2}\delta_{s,r}           & - \frac{\vec{k}\vec{\sigma}}{E_{\vec{k}} + m} \mathbb{1}_{2\times2}\delta_{s,r} \\
		\frac{\vec{k}\vec{\sigma}}{E_{\vec{k}} + m}\mathbb{1}_{2\times2}\delta_{s,r} & -\mathbb{1}_{2\times2}\delta_{s,r}
	\end{pmatrix}                                                                                                                                                                           \\
	 & =
	\begin{pmatrix}
		\frac{k^2}{E_{\vec{k}} + m} & -\vec{k}\vec{\sigma} \\
		\vec{k}\vec{\sigma}         & -E_{\vec{k}} - m
	\end{pmatrix}
	\otimes \mathbb{1}_{2\times2}                                                                                                                                                                                                                                                                                                                                                                          \\
	 & = E_{\vec{k}}
	\begin{pmatrix}
		1 & 0  \\
		0 & -1
	\end{pmatrix} \otimes \mathbb{1}_{2\times2}
	- m\begin{pmatrix}
		   1 & 0 \\
		   0 & 1
	   \end{pmatrix} \otimes \mathbb{1}_{2\times2}
	+ \vec{k} \begin{pmatrix}
		          0            & -\vec{\sigma} \\
		          \vec{\sigma} & 0
	          \end{pmatrix}\otimes \mathbb{1}_{2\times2}                                                                                                                                                                                                                                                                                                                                                   \\
	 & = E_{\vec{k}} \gamma^0 - \vec{k}\vec{\gamma} - m                                                                                                                                                                                                                                                                                                                                                    \\
	 & = \slashed{k} - m
\end{align*}

Chứng minh $\bar{v}_{k}^{(s)} v^{(r)}_{k} = -2m\delta_{s,r} $
\begin{align*}
	RHS 
	& =
	\sqrt{E_{\vec{k}} + m }
	\begin{pmatrix}
		\frac{\vec{k}\vec{\sigma}}{E_{\vec{k}} + m } \chi^{(s)^{\dagger} } & \chi^{(s)^{\dagger} }
	\end{pmatrix}\gamma^0
	\sqrt{E_{\vec{k}} + m }
	\begin{pmatrix}
		\frac{\vec{k}\vec{\sigma}}{E_{\vec{k}} + m } \chi^{(r)} \\
		\chi^{(r)}
	\end{pmatrix}\\
	&= (E_{\vec{k}}+m)
	\begin{pmatrix}
		\frac{\vec{k}\vec{\sigma}}{E_{\vec{k}} + m } \chi^{(s)^{\dagger} } & -\chi^{(s)^{\dagger} }
	\end{pmatrix}
	\begin{pmatrix}
		\frac{\vec{k}\vec{\sigma}}{E_{\vec{k}} + m } \chi^{(r)} \\
		\chi^{(r)}
	\end{pmatrix}\\
	& = (E_{\vec{k}}+m)
		\left[\frac{\vec{k}\vec{\sigma}}{E_{\vec{k}} + m } \chi^{(s)^{\dagger}}\frac{\vec{k}\vec{\sigma}}{E_{\vec{k}} + m } \chi^{(r)} -\chi^{(s)^{\dagger}} \chi^{(r)}\right] \\
	& = \left[ \frac{k^2}{E_{\vec{k}} + m}\delta_{s,r} - (E_{\vec{k}} + m)\delta_{s,r}\right] \\
	& = \left[ (E_{\vec{k}} - m)\delta_{s,r} - (E_{\vec{k}} + m)\delta_{s,r}\right] \\
	& = -2m \delta_{s,r} (\text{ĐPCM}) \\
\end{align*}

$\displaystyle\dv[n]{f}{x}$
$\dfrac{df}{dx}$

\end{document}