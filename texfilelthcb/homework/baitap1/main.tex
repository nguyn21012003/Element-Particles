\documentclass{report}
\usepackage[utf8]{vietnam}
\usepackage[utf8]{inputenc}
\usepackage{anyfontsize,fontsize}
\changefontsize[13pt]{13pt}
\usepackage{commath}
\usepackage{blindtext}
\usepackage{parskip}
\usepackage{xcolor}
\usepackage{amssymb}
\usepackage{slashed}
\usepackage{indentfirst}
\usepackage{pdfpages}
\usepackage{graphicx}

\usepackage{tikz-feynman}
\usepackage{nccmath}
\usepackage{mathtools}
\usepackage{amsfonts}
\usepackage{amsmath,systeme,bbold}
\usepackage[thinc]{esdiff}
\usepackage{hyperref}
\usepackage{dirtytalk,bm,physics,cancel}
\usepackage{tikz}
\usepackage{lipsum}
\usepackage{fancyhdr}
%footnote
\pagestyle{fancy}
\renewcommand{\headrulewidth}{0pt}%
\fancyhf{}%
\fancyfoot[L]{Vật lý Lý thuyết}%
\fancyfoot[C]{\hspace{4cm} \thepage}%

\usetikzlibrary{shapes.geometric, arrows}

\usepackage{geometry}
\geometry{
	a4paper,
	total={170mm,257mm},
	left=20mm,
	top=20mm,
}

\renewcommand{\baselinestretch}{2.0}
\usetikzlibrary{arrows.spaced}
\usetikzlibrary{animations,quotes}
%gian do
\tikzstyle{startstop} = [rectangle, rounded corners, minimum width=3cm, minimum height=1cm, text centered,draw=black, fill=white!30]
\tikzstyle{arrow} = [thick,->,>=stealth]

\title{\Huge{Bài tập 1}}

\hypersetup{
	colorlinks=true,
	linkcolor=red,
	filecolor=magenta,      
	urlcolor=cyan,
	pdftitle={},
	pdfpagemode=FullScreen,
}

\urlstyle{same}

\begin{document}
\setlength{\parindent}{20pt}
\newpage
\author{TRẦN KHÔI NGUYÊN \\ VẬT LÝ LÝ THUYẾT}
\maketitle
Giải:\\
Cho:
\begin{align*}
	\begin{cases}
		\phi(\vec{x}) & = \displaystyle \int \dfrac{d^3\vec{k}}{(2\pi)^{3/2}} \dfrac{1}{\sqrt{2\omega_{\vec{k}}}}\left[ a_{\vec{k}}e^\alpha + a_{\vec{k}}^\dagger e^{-\alpha} \right]     \\
		\pi(\vec{x})  & = (-i)\displaystyle \int \dfrac{d^3\vec{k}}{(2\pi)^{3/2}} \sqrt{\dfrac{\omega_{\vec{k}}}{2}} \left[ a_{\vec{k}}e^\alpha + a_{\vec{k}}^\dagger e^{-\alpha} \right] \\
	\end{cases}
\end{align*}
trong đó $\alpha = -i\omega_{\vec{k}} t + i\vec{k}\cdot\vec{x}$.

Phép biến đổi Fourier 4D cho $\phi(\vec{x})$ và $\pi(\vec{x})$:
\begin{align*}
	\ast\int d^3x \phi(\vec{x}) e^{i\vec{k'}\cdot\vec{x}} & = \int d^3x \int \dfrac{d^3\vec{k}}{(2\pi)^{3/2}} \dfrac{1}{\sqrt{2\omega_{\vec{k}}}}\left[ a_{\vec{k}}e^\alpha + a_{\vec{k}}^\dagger e^{-\alpha} \right] e^{i\vec{k'}\cdot\vec{x}}                                                                                                        \\
	                                                      & = \int d^3x \int \dfrac{d^3\vec{k}}{(2\pi)^{3/2}} \dfrac{1}{\sqrt{2\omega_{\vec{k}}}} \left[ a_{\vec{k}}e^\alpha e^{i\vec{k'}\cdot\vec{x}}  + a_{\vec{k}}^\dagger e^{-\alpha}e^{i\vec{k'}\cdot\vec{x}} \right]                                                                             \\
	                                                      & = \int d^3x \int \dfrac{d^3\vec{k}}{(2\pi)^{3/2}} \dfrac{1}{\sqrt{2\omega_{\vec{k}}}} \left[ a_{\vec{k}}e^{ -i\omega_{\vec{k}} t + i\vec{k}\cdot\vec{x}} e^{i\vec{k'}\cdot\vec{x}}  + a_{\vec{k}}^\dagger e^{ i\omega_{\vec{k}} t - i\vec{k}\cdot\vec{x}}e^{i\vec{k'}\cdot\vec{x}} \right] \\
	                                                      & = (2\pi)^3 \int \dfrac{d^3\vec{k}}{(2\pi)^{3/2}} \dfrac{1}{\sqrt{2\omega_{\vec{k}}}} \left[ a_{\vec{k}}e^{-i\omega_{\vec{k}}t} \delta(\vec{k}+\vec{k'}) + a_{\vec{k}}^\dagger e^{i\omega_{\vec{k}}t}\delta(\vec{k'}-\vec{k}) \right],
\end{align*}
do $\displaystyle \int d^3x e^{i(\vec{k'}+\vec{k})\cdot x} = (2\pi)^3\delta(\vec{k}+\vec{k'})$ và $\displaystyle \int d^3x e^{i(\vec{k'}-\vec{k})\cdot x} = (2\pi)^3\delta(\vec{k}-\vec{k'})$
\begin{align}
	&\Rightarrow \int d^3x \phi(\vec{x}) e^{i\vec{k'}\cdot\vec{x}}
	  = \dfrac{(2\pi)^{\frac{3}{2}}}{\sqrt{2\omega_{\vec{k'}}}}\left[ a_{-\vec{k'}}e^{-i\omega_{\vec{k'}}t} + a_{\vec{k'}}^{\dagger}e^{i\omega_{\vec{-k'}}t} \right]
	= \dfrac{(2\pi)^{\frac{3}{2}}}{\sqrt{2\omega_{\vec{q}}}}\left[ a_{-\vec{q}}\,e^{-i\omega_{\vec{q}}t} + a_{\vec{q}}^{\dagger}\,e^{i\omega_{-\vec{q}}t} \right]  \nonumber \\
	&\Rightarrow \dfrac{\sqrt{2\omega_{\vec{q}}}}{(2\pi)^{\frac{3}{2}}} \int d^3x \phi(\vec{x}) e^{i\vec{k'}\cdot\vec{x}}
	  = \left[ -a_{\vec{q}}\,e^{-i\omega_{\vec{q}}t} + a_{\vec{q}}^{\dagger}\,e^{i\omega_{-\vec{q}}t} \right] \label{eq1}
\end{align}

\begin{align*}
	\ast\int d^3x \pi(\vec{x}) e^{i\vec{k'}\cdot\vec{x}} & = \int d^3x (-i)\int \dfrac{d^3\vec{k}}{(2\pi)^{3/2}} \sqrt{\dfrac{\omega_{\vec{k}}}{2}} \left[ a_{\vec{k}}e^\alpha - a_{\vec{k}}^\dagger e^{-\alpha} \right] e^{i\vec{k'}\cdot\vec{x}}                                                                                                       \\
	                                                     & = \int d^3x (-i)\int \dfrac{d^3\vec{k}}{(2\pi)^{3/2}} \sqrt{\dfrac{\omega_{\vec{k}}}{2}} \left[ a_{\vec{k}}e^\alpha e^{i\vec{k'}\cdot\vec{x}}  - a_{\vec{k}}^\dagger e^{-\alpha}e^{i\vec{k'}\cdot\vec{x}} \right]                                                                             \\
	                                                     & = (-i)\int d^3x \int \dfrac{d^3\vec{k}}{(2\pi)^{3/2}} \sqrt{\dfrac{\omega_{\vec{k}}}{2}} \left[ a_{\vec{k}}e^{ -i\omega_{\vec{k}} t + i\vec{k}\cdot\vec{x}} e^{i\vec{k'}\cdot\vec{x}}  - a_{\vec{k}}^\dagger e^{ i\omega_{\vec{k}} t - i\vec{k}\cdot\vec{x}}e^{i\vec{k'}\cdot\vec{x}} \right] \\
	                                                     & = -i(2\pi)^3 \int \dfrac{d^3\vec{k}}{(2\pi)^{3/2}} \sqrt{\dfrac{\omega_{\vec{k}}}{2}} \left[ a_{\vec{k}}e^{-i\omega_{\vec{k}}t} \delta(\vec{k}+\vec{k'}) - a_{\vec{k}}^\dagger e^{i\omega_{\vec{k}}t}\delta(\vec{k'}-\vec{k}) \right],
\end{align*}
do $\displaystyle \int d^3x e^{i(\vec{k'}+\vec{k})\cdot x} = (2\pi)^3\delta(\vec{k}+\vec{k'})$ và $\displaystyle \int d^3x e^{i(\vec{k'}-\vec{k})\cdot x} = (2\pi)^3\delta(\vec{k}-\vec{k'})$
\begin{align}
	&\Rightarrow\int d^3x \pi(\vec{x}) e^{i\vec{k'}\cdot\vec{x}}
	  = -i(2\pi)^{\frac{3}{2}}\sqrt{\dfrac{\omega_{\vec{k}}}{2}}\left[ a_{-\vec{k'}}e^{-i\omega_{\vec{k'}}t} - a_{\vec{k'}}^{\dagger}e^{i\omega_{\vec{-k'}}t} \right]\nonumber\\
	&=-i(2\pi)^{\frac{3}{2}}\sqrt{\dfrac{\omega_{\vec{q}}}{2}}\left[ a_{-\vec{q}}\,e^{-i\omega_{\vec{q}}t} - a_{-\vec{q}}^{\dagger}\,e^{i\omega_{-\vec{q}}t} \right]\nonumber \\
	 &\Rightarrow i\sqrt{\dfrac{2}{\omega_{\vec{q}}}}\dfrac{1}{(2\pi)^\frac{3}{2}}\int d^3x \pi(\vec{x}) e^{i\vec{k'}\cdot\vec{x}}
	  = \left[ a_{-\vec{q}}\,e^{-i\omega_{\vec{q}}t} - a_{\vec{q}}^{\dagger}\,e^{i\omega_{-\vec{q}}t} \right] \label{eq2}
\end{align}
Từ \hyperref[eq1]{(1)} và \hyperref[eq2]{(2)}:
\begin{align*}
	\begin{cases}
		 a_{-\vec{q}}\,e^{-i\omega_{\vec{q}}t} + a_{\vec{q}}^{\dagger}\,e^{i\omega_{-\vec{q}}t} = \dfrac{\sqrt{2\omega_{\vec{q}}}}{(2\pi)^{\frac{3}{2}}} \displaystyle \int d^3x \phi(\vec{x}) e^{i\vec{k'}\cdot\vec{x}},\\
	a_{-\vec{q}}\,e^{-i\omega_{\vec{q}}t} - a_{\vec{q}}^{\dagger}\,e^{i\omega_{-\vec{q}}t}  = i\sqrt{\dfrac{2}{\omega_{\vec{q}}}}\dfrac{1}{(2\pi)^\frac{3}{2}} \displaystyle\int d^3x \pi(\vec{x}) e^{i\vec{k'}\cdot\vec{x}},
	\end{cases}
\end{align*}
lấy \hyperref[eq1]{(1)} + \hyperref[eq2]{(2)}:
\begin{align}
	 2 a_{-\vec{q}}\,e^{-i\omega_{\vec{q}}t} 
	 &= \dfrac{\sqrt{2\omega_{\vec{q}}}}{(2\pi)^{\frac{3}{2}}} \displaystyle \int d^3x \phi(\vec{x}) e^{i\vec{k'}\cdot\vec{x}} + i\sqrt{\dfrac{2}{\omega_{\vec{q}}}}\dfrac{1}{(2\pi)^\frac{3}{2}} \displaystyle\int d^3x \pi(\vec{x}) e^{i\vec{k'}\cdot\vec{x}} \nonumber \\
	 \Rightarrow  a_{-\vec{q}} &= \dfrac{1}{2(2\pi)^\frac{3}{2}}  \displaystyle \int d^3x \left[\sqrt{2\omega_{\vec{q}}}\,\phi(\vec{x}) + i\sqrt{\dfrac{2}{\omega_{\vec{q}}}}\,\pi(\vec{x})\right] e^{i\vec{k'}\cdot\vec{x}} e^{i\omega_{\vec{q}}t}, \label{eq3} 
\end{align}
lấy \hyperref[eq1]{(1)} - \hyperref[eq2]{(2)}:
\begin{align}
	2 a_{-\vec{q}}\,e^{i\omega_{\vec{q}}t} 
	&= \dfrac{\sqrt{2\omega_{\vec{q}}}}{(2\pi)^{\frac{3}{2}}} \displaystyle \int d^3x \phi(\vec{x}) e^{i\vec{k'}\cdot\vec{x}} - i\sqrt{\dfrac{2}{\omega_{\vec{q}}}}\dfrac{1}{(2\pi)^\frac{3}{2}} \displaystyle\int d^3x \pi(\vec{x}) e^{i\vec{k'}\cdot\vec{x}} \nonumber \\
	\Rightarrow  a_{-\vec{q}}^{\dagger} &= \dfrac{1}{2(2\pi)^\frac{3}{2} }  \displaystyle \int d^3x \left[\sqrt{2\omega_{\vec{q}}}\,\phi(\vec{x}) - i\sqrt{\dfrac{2}{\omega_{\vec{q}}}}\,\pi(\vec{x})\right] e^{i\vec{k'}\cdot\vec{x}} e^{-i\omega_{\vec{q}}t}, \label{eq4} 
\end{align}
từ \hyperref[eq3]{(3)} và \hyperref[eq4]{(4)}:
\begin{equation}
	\begin{cases}
		\begin{aligned}
			a_{\vec{q}}^{\dagger} &= \dfrac{1}{2(2\pi)^\frac{3}{2} }  \displaystyle \int d^3x \left[\sqrt{2\omega_{\vec{q}}}\,\phi(\vec{x}) + i\sqrt{\dfrac{2}{\omega_{\vec{q}}}}\,\pi(\vec{x})\right] e^{i\vec{k'}\cdot\vec{x}} e^{i\omega_{\vec{q}}t}\\
			a_{\vec{q}}^{\dagger} &= \dfrac{1}{2(2\pi)^\frac{3}{2} }  \displaystyle \int d^3x \left[\sqrt{2\omega_{\vec{q}}}\,\phi(\vec{x}) - i\sqrt{\dfrac{2}{\omega_{\vec{q}}}}\,\pi(\vec{x})\right] e^{i\vec{k'}\cdot\vec{x}} e^{-i\omega_{\vec{q}}t}
		\end{aligned}
	\end{cases}\nonumber
\end{equation}
thay $k'=-k'$ cho \hyperref[eq3]{(3)}
\begin{equation}
	\Rightarrow
	\begin{cases}
		\begin{aligned}
			a_{\vec{q}} &= \dfrac{1}{2(2\pi)^\frac{3}{2} }  \displaystyle \int d^3x \left[\sqrt{2\omega_{\vec{q}}}\,\phi(\vec{x}) + i\sqrt{\dfrac{2}{\omega_{\vec{q}}}}\,\pi(\vec{x})\right] e^{i(\omega_{\vec{q}}t - \vec{k'}\cdot\vec{x})}\\
			a_{\vec{q}}^{\dagger} &= \dfrac{1}{2(2\pi)^\frac{3}{2} }  \displaystyle \int d^3x \left[\sqrt{2\omega_{\vec{q}}}\,\phi(\vec{x}) - i\sqrt{\dfrac{2}{\omega_{\vec{q}}}}\,\pi(\vec{x})\right] e^{-i(\omega_{\vec{q}}t - \vec{k'}\cdot\vec{x})}
		\end{aligned}
	\end{cases}\nonumber
\end{equation}

$\ast$ Tính các giao hoán tử:
\begin{align*}
	\left[	a_{-\vec{q}} \,, a_{\vec{q}}^{\dagger}\right] &= a_{-\vec{q}}\, a_{\vec{q}}^{\dagger} - a_{\vec{q}}^{\dagger}\, a_{-\vec{q}}
\end{align*}
Để thuận tiện trong việc đọc các kí hiệu, thì ta sẽ thay đổi một số dummy variables:
\begin{align*}
	a_{\vec{k}} &= \dfrac{1}{2(2\pi)^\frac{3}{2} }  \displaystyle \int d^3x \left[\sqrt{2\omega_{\vec{k}}}\,\phi(\vec{x}) + i\sqrt{\dfrac{2}{\omega_{\vec{k}}}}\,\pi(\vec{x})\right] e^{i(\omega_{\vec{k}}t - \vec{k}\cdot\vec{x})}\;\text{thay $-q = k$ và $k' = k$},
\end{align*}
và ``tạm'' bỏ đi các kí hiệu vector. Giao hoán tử sẽ trở thành:
\begin{align*}
	\left[	a_{\vec{k}} \,, a_{\vec{q}}^{\dagger}\right] 
	&= a_{\vec{k}}\, a_{\vec{q}}^{\dagger} - a_{\vec{q}}^{\dagger}\, a_{\vec{k}}\\
	&= \dfrac{1}{2} \int \dfrac{d^3xd^3y}{(2\pi^3)}\dfrac{e^{i(\omega_k - \omega_q)t} e^{-i(k\cdot x - q\cdot y)}}{\sqrt{\omega_k\omega_q}}  \\
	&\times \biggl[ \omega_k \omega_q \phi(x)\phi(y) - i\omega_k\phi(x)\pi(y) + i\omega_q\pi(x)\phi(y) + \pi(x)\pi(y) \\
	&- \omega_k \omega_q\phi(y)\phi(x) + i\omega_k\pi(y)\phi(x) - i\omega_q\phi(y)\pi(x) - \pi(y)\pi(x) \biggr]\\
	& = \dfrac{1}{2} \int \dfrac{d^3xd^3y}{(2\pi^3)}\dfrac{e^{i(\omega_k - \omega_q)t} e^{-i(k\cdot x - q\cdot y)}}{\sqrt{\omega_k\omega_q}}\biggl\{ \omega_k\omega_q\left[ \phi(x),\phi(y) \right] - i\omega_k\left[\phi(x),\pi(y)\right]\\
	&- i\omega_q\left[\phi(y),\pi(x)\right] + \left[ \pi(x),\pi(y) \right] \biggr\}\\
	& = \dfrac{1}{2} \int \dfrac{d^3xd^3y}{(2\pi^3)}\dfrac{e^{i(\omega_k - \omega_q)t} e^{-i(k\cdot x - q\cdot y)}}{\sqrt{\omega_k\omega_q}}\biggl\{ \cancel{\omega_k\omega_q\left[ \phi(x),\phi(y) \right]} - i\omega_k\left[\phi(x),\pi(y)\right]\\
	&- i\omega_q\left[\phi(y),\pi(x)\right] + \cancel{\left[ \pi(x),\pi(y) \right] \biggr\}}
\end{align*}
số hạng thứ 1 và số hạng thứ 4 bị biến mất là do $\left[A(x),A(y)\right] = 0$, nên:
\begin{align*}
	& = \dfrac{1}{2} \int \dfrac{d^3xd^3y}{(2\pi^3)}\dfrac{e^{i(\omega_k - \omega_q)t} e^{-i(k\cdot x - q\cdot y)}}{\sqrt{\omega_k\omega_q}} (-i)\left[\omega_k i \delta(x-y) + \omega_q i \delta(y-x)  \right] \\ 
	&= 	\dfrac{e^{i(\omega_k - \omega_q)t}}{2(2\pi)^3} \dfrac{\omega_k + \omega_q}{\sqrt{\omega_k\omega_q}} \int d^3x e^{i(q-k)x}\\
	&= 	\dfrac{e^{i(\omega_k - \omega_q)t}}{2(2\pi)^3} \dfrac{\omega_k + \omega_q}{\sqrt{\omega_k\omega_q}} (2\pi)^3 \delta(q-k)\\
	&= f(k,q)\delta(q-k ) = \delta(q-k) \; \text{do khi $k=q$ thì $f(k,q) \rightarrow 1$ }
\end{align*}






























\end{document}