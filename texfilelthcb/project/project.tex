\documentclass{report}
\usepackage[utf8]{vietnam}
\usepackage[utf8]{inputenc}
\usepackage{anyfontsize,fontsize}
\changefontsize[13pt]{13pt}	
\usepackage{commath}
\usepackage[d]{esvect}
\usepackage{parskip}
\usepackage{xcolor}
\usepackage{amssymb}
\usepackage{slashed}
\usepackage{indentfirst}
\usepackage{pdfpages}
\usepackage{graphicx}
\usepackage{nccmath}
\usepackage{mathtools}
\usepackage{amsfonts}
\usepackage{amsmath,systeme,nicematrix}
\usepackage[thinc]{esdiff}
\usepackage{hyperref}
\usepackage{bm,physics}
\usepackage{fancyhdr}
%footnote
\pagestyle{fancy}
\renewcommand{\headrulewidth}{0pt}%
\fancyhf{}%
\fancyfoot[L]{Vật lý Lý thuyết}%
\fancyfoot[C]{\hspace{4cm} \thepage}%


\usepackage{geometry}
\geometry{
	a4paper,
	total={170mm,257mm},
	left=20mm,
	top=20mm,
}


\newcommand{\image}[1]{
	\begin{center}
		\includegraphics[width=0.5\textwidth]{pic/#1}
	\end{center}
}
\renewcommand{\l}{\ell}
\newcommand{\dps}{\displaystyle}

\newcommand{\f}[2]{\dfrac{#1}{#2}}
\newcommand{\at}[2]{\bigg\rvert_{#1}^{#2} }


\renewcommand{\baselinestretch}{2.0}


\title{\Huge{Hạt cơ bản}}

\hypersetup{
	colorlinks=true,
	linkcolor=red,
	filecolor=magenta,      
	urlcolor=cyan,
	pdftitle={hcb},
	pdfpagemode=FullScreen,
}

\urlstyle{same}

\begin{document}
\setlength{\parindent}{20pt}
\newpage
\author{TRẦN KHÔI NGUYÊN \\ VẬT LÝ LÝ THUYẾT}
\maketitle
\newpage
\subsection*{Review Standard Model}
\subsubsection*{Four-vector \& hàm sóng}
Vector phản biến
\begin{align*}
	x^{\mu} = (t,x,y,z) = (t,\mathbf{x}) \\
	p^{\mu} = (E, p_x , p_y , p_z) = (E , \mathbf{p}).
\end{align*}

Vector hiệp biến
\begin{align*}
	x_{\mu} &= (t,-x,-y,-z) = (t,-\mathbf{x}) \\
	p_{\mu} &= (E, -p_x , -p_y , -p_z) = (E , -\mathbf{p}).
\end{align*}

Đạo hàm hiệp biến
\begin{align*}
	\partial_{\mu} = \left( \f{\partial}{\partial t}; \f{\partial}{\partial x} , \f{\partial}{\partial y} , \f{\partial}{\partial z} \right)
\end{align*}

Hàm sóng Fermion
\begin{align*}
	\psi(\mathbf{x},t) = u (E,\mathbf{p}) e^{(i\mathbf{px} -Et)} = u(p) e^{-i p x}.
\end{align*}
\subsubsection*{Trường}
\begin{enumerate}
	\item[(a)] Trường vô hướng: Xét $\Phi(x)$ là trường vô hướng thực. Ta thực hiện phép biến đổi Lorentz:
	\begin{align*}
		\mathcal{L} =: x^{\mu} \rightarrow x^{'\mu} = \Delta_{\nu}^{\mu}x^{\nu}.
	\end{align*} 
	Và biểu diễn của nhóm Lorentz tác động lên trường vô hướng thực:
	\begin{align*}
		\Phi(x) = \Phi_{a}^{'}(x') = M_{ab}(\Delta) \Phi_{b}(x).
	\end{align*}
	Trong đó, biểu diễn nhóm $M_{ab}(\Delta)$ có dạng:
	\begin{itemize}
		\item $I$ đối với trường vô hướng (scarlar field $s = 0$).
		\item $e^{\frac{1}{4} \sum^{\mu\nu} \omega_{\mu\nu}}$ với $\sum^{\mu\nu} = \left[\gamma^{\mu}, \gamma^{\nu} \right]$ đối với trường spinor ($s = \f{1}{2}$).
		\item  $\Delta_{\nu}^{\mu}$ đối với trường vector ($s = 1,3/2 ,2 ,..$)
	\end{itemize}
	Lagrangian có dạng:
	\begin{align*}
		\mathcal{L} = \f{1}{2} \partial^{\mu} \Phi \partial_{\mu} \Phi - \f{1}{2} m^{2} \Phi^{2} \tag{1}
	\end{align*}
	và có thể có các số hạng tự tương tác bậc cao $\alpha \Phi^{3} ,...$\\
	Phương trình chuyển động(Phương trình Euler-Lagrange)
	\begin{align*}
		\f{\partial\mathcal{L}}{\partial\Phi} -\partial_{\mu} \left( \f{\partial \mathcal{L}}{\partial(\partial \Phi)} \right) = 0.
	\end{align*}
	Phương trình Klein-Gordon.
	\begin{align*}
		(\square + m^{2}) \Phi = 0
	\end{align*}
	có nghiệm là
	\begin{align*}
		\Phi = \int \f{d^{3} \mathbf{k}}{(2\pi)^{3/2}} \f{1}{\sqrt{2\omega+{\vv{k}}}} \left( a_{\vv{k}}e^{ -i \omega_{\vv{k}t + i \vv{k}\vv{x}}} + a_{\vv{k}}^{\ast} e^{ i \omega_{\vv{k}t - i \vv{k}\vv{x}} } \right)
	\end{align*}
	Lượng tử hóa, định nghĩa xung lượng liên hợp
	\begin{align*}
		\pi(x) = \dot{\phi}(x) = i \int \f{d^{3}}{(2\pi)^{3}} \sqrt{\f{\omega_{\vv{k}}}{2}} \left( - a_{\vv{k}} e^{ -i \omega_{\vv{k}} t + i \vv{k} \vv{x} } + a_{\vv{k}}^{*} e^{ i \omega_{\vv{k}} t - i \vv{k} \vv{x} } \right)
	\end{align*}
	Giao hoán tử
	\begin{align*}
		\left[\Phi,\Pi\right] = i \delta(\vv{x} - \vv{y})
	\end{align*}
	\item [(b)] Trường vector\\
	Hàm trường
	\begin{align*}
		A_{\mu} = \phi_{,} - \vv{A} \quad \quad ; A^{\mu} = \left( \phi , \vv{A}\right).
	\end{align*}
	Tensor trường vector:
	\begin{align*}
		F_{\mu\nu} = \partial_{\mu} A_{\nu} - \partial_{\nu} A_{\mu}.
	\end{align*}
	Lagrangian 
	\begin{align*}
		\mathcal{L} = -\f{1}{4} F_{\mu\nu} F^{\mu\nu}.
	\end{align*}
	Phương trình chuyển động
	\begin{align*}
		\partial_{\rho} F^{\sigma} = 0
	\end{align*}
	\item[c] Trường spinor\\
	Hàm trường
	\begin{align*}
		\psi(x) =
		\begin{pNiceMatrix}
			\psi(x)_{1}\\
			\psi(x)_{2}\\
			\psi(x)_{3}\\
			\psi(x)_{4}
		\end{pNiceMatrix}
	\end{align*}
	Lagrangian
	\begin{align*}
		\mathcal{L} = i \bar{\psi} \gamma^{\mu} \partial_{\mu} \psi - m \bar{\psi} \psi.
	\end{align*}
\end{enumerate}
\subsubsection*{Toán tử chiếu}
\begin{align*}
	P_{L} = \f{1 - \gamma^{5}}{2}\\	
	P_{R} = \f{1 + \gamma^{5}}{2}	
\end{align*}
với các tính chất
\begin{itemize}
	\item $P^{2} = P$
	\item $P_{L} + P_{R} = 1$
	\item $P_{L}P_{R} = 0$
\end{itemize}
\subsubsection*{Nhóm Gauge}
\begin{align*}
	SU(3)_{C} \times SU(2)_{L} \times U(1)_{Y} 
\end{align*}
trong đó
\begin{itemize}
	\item $SU(3)_{C}$ là các tương tác mạnh thông qua 8 gluon $G^{a}$($a = 1..8$)
	\item $SU(2)$ lầ các tương tác yếu thông qua 3 gauge bosons
	\begin{align*}
		W^{\pm} = \f{1}{\sqrt{2}} (W^{1} \mp i W^{2}) ; \quad \quad W^{3}
	\end{align*}
	\item $U(1)$ là các tương tác điện từ thông qua bởi duy một gauge boson là $B$
\end{itemize}
\subsubsection*{Thế hệ hạt}
\begin{align*}
\begin{array}{l@{\hspace{1.5cm}}l@{\hspace{1.5cm}}l@{\hspace{1.5cm}}l}
	\text{first generation} & \text{second generation} & \text{third generation} & \text{each quark} \\[0.2cm]
	\left(
	\begin{array}{c}
		u_L \\ d_L
	\end{array}
	\right)
	\hspace{0.2cm}
	\begin{array}{l}
		u_R \\
		d_R
	\end{array}
	&
	\left(
	\begin{array}{c}
		c_L \\ s_L
	\end{array}
	\right)
	\hspace{0.2cm}
	\begin{array}{l}
		c_R \\
		s_R
	\end{array}
	&
	\left(
	\begin{array}{c}
		t_L \\ b_L
	\end{array}
	\right)
	\hspace{0.2cm}
	\begin{array}{l}
		t_R \\
		b_R
	\end{array}
	&
	q = \left(
	\begin{array}{c}
		q_r \\ q_g \\ q_b
	\end{array}
	\right) \\[0.5cm]
	\left(
	\begin{array}{c}
		\nu_{eL} \\ e_L
	\end{array}
	\right)
	\hspace{0.2cm}
	\begin{array}{l}
		\nu_{eR}(?) \\
		e_R
	\end{array}
	&
	\left(
	\begin{array}{c}
		\nu_{\mu L} \\ \mu_L
	\end{array}
	\right)
	\hspace{0.2cm}
	\begin{array}{l}
		\nu_{\mu R}(?) \\
		\mu_R
	\end{array}
	&
	\left(
	\begin{array}{c}
		\nu_{\tau L} \\ \tau_L
	\end{array}
	\right)
	\hspace{0.2cm}
	\begin{array}{l}
		\nu_{\tau R}(?) \\
		\tau_R
	\end{array}
	&
\end{array}
\end{align*}
\subsubsection*{Lagrangian của mô hình chuẩn}
\begin{align*}
	\mathcal{L}
	& = i \overline{L}^{i} \slashed{D} L^{i} + i \overline{Q_{L}}^{i} \slashed{D} Q_{L}^{i} + i \overline{E_{R}}^{i} \slashed{D} E_{R}^{i} + i \overline{U_{L}}^{i} \slashed{D} U_{L}^{i} + i \overline{D_{L}}^{i} \slashed{D} D_{L}^{i} \\
	& - \f{1}{4}G^{a \, \mu\nu}G_{\mu\nu}^{a} - \f{1}{4}W^{a \, \mu\nu}W_{\mu\nu}^{a} - \f{1}{4}B^{\mu\nu}B_{\mu\nu}.
\end{align*}
trong đó
\begin{align*}
	D_{\mu}(f) 
	&= \partial_{\mu} - i g \left( T^{+} W_{\mu}^{+} + T^{-} W_{\mu}^{-} \right) - \f{ig}{C_{w}} \left[ I_{3}^{(f)} - S_{w}^{2} Q^{(f)} \right] Z_{\mu} - ie Q^{(f)} A_{\mu}\\
	F_{\mu\nu}^{a} &= \partial_{\mu} A_{\nu}^{a} - \partial_{\nu} A_{\mu}^{a} + g f^{abc} A_{\mu}^{b} A_{\nu}^{c}
\end{align*}

\subsection*{Chứng minh công thức đạo hàm kéo dài}
\begin{align*}
	D_{\mu}(f) = \partial_{\mu} - i g \left( T^{+} W_{\mu}^{+} + T^{-} W_{\mu}^{-} \right) - \f{ig}{C_{w}} \left[ I_{3}^{(f)} - S_{w}^{2} Q^{(f)} \right] Z_{\mu} - ie Q^{(f)} A_{\mu}
\end{align*}
trong đó
\begin{align*}
	T^{+} =
	\begin{pNiceMatrix}
		0 & 1 \\
		0 & 0
	\end{pNiceMatrix};\quad \quad
	T^{-} =
	\begin{pNiceMatrix}
		0 & 0 \\
		1 & 0
	\end{pNiceMatrix}
\end{align*}
Ta xét đạo hàm hiệp biến dưới phép biến đổi sau:
\begin{align*}
	\begin{cases}
		W_{\mu}^{\pm} = \f{W_{\mu}^{1} \mp i W_{\mu}^{2} }{\sqrt{2}} \\
		\begin{pNiceMatrix}
			Z_{\mu} \\
			A_{\mu}
		\end{pNiceMatrix}
		=
		\begin{pNiceMatrix}
			C_{W} & -S_{W} \\
			S_{W} & C_{W}
		\end{pNiceMatrix}
		\begin{pNiceMatrix}
			W_{\mu}^{3} \\
			B_{\mu}
		\end{pNiceMatrix}
	\end{cases}
\end{align*}
Ta có
\begin{align*}
	D_{\mu}
	 & = \partial_{\mu} - ig \f{\sigma^{a}}{2} W_{\mu}^{a} - ig'\f{Y}{2}B_{\mu}                                                                \\
	 & = \partial_{\mu} - \f{ig}{2} \left( \sigma^{1}W_{\mu}^{1} + \sigma^{2}W_{\mu}^{2} + \sigma^{3}W_{\mu}^{3} \right) - ig' \f{Y}{2}B_{\mu} \\
	 & = \partial_{\mu} - \f{ig}{2}
	\begin{pNiceMatrix}
		0                           & W_{\mu}^{1} - i W_{\mu}^{2} \\
		W_{\mu}^{1} + i W_{\mu}^{2} & 0
	\end{pNiceMatrix}
	- \f{ig}{2} \sigma^{3}W_{\mu}^{3} - ig' \f{Y}{2} B_{\mu}                                                                                   \\
	 & = \partial_{\mu} - \f{ig}{\sqrt{2}}
	\begin{pNiceMatrix}
		0           & W_{\mu}^{+} \\
		W_{\mu}^{-} & 0
	\end{pNiceMatrix}
	- \f{ig}{2}\sigma^{3}W_{\mu}^{3} - ig' \f{Y}{2} B_{\mu} \tag{1}
\end{align*}
mà
\begin{align*}
	\begin{cases}
		g' = g t_{W} \\
		T_{3}L^{i} = \f{\sigma^{3}}{2}L^{i}
	\end{cases}
\end{align*}
thay vào (1) ta được
\begin{align*}
	D_{\mu} = \partial_{\mu} - \f{ig}{\sqrt{2}}
	\Bigg[
		\begin{pNiceMatrix}
			0 & 1 \\
			0 & 0
		\end{pNiceMatrix}
		W_{\mu}^{+}
		+
		\begin{pNiceMatrix}
			0 & 0 \\
			1 & 0
		\end{pNiceMatrix}
		W_{\mu}^{-}
		\Bigg]
	-ig I_{3}\left( C_{W} Z_{\mu} + S_{W} A_{\mu} \right) - ig t_{W} \f{Y}{2} \left( C_{W} A_{\mu} - S_{W} Z_{\mu} \right)
\end{align*}
với
\begin{align*}
	T^{+}
	\begin{pNiceMatrix}
		0 & 1 \\
		0 & 0
	\end{pNiceMatrix} \\
	T^{-}
	\begin{pNiceMatrix}
		0 & 0 \\
		1 & 0
	\end{pNiceMatrix}
\end{align*}
\begin{align*}
	\rightarrow D_{\mu} = \partial_{\mu} - \f{ig}{\sqrt{2}} \left( T^{+} W_{\mu}^{+} + T^{-} W_{\mu}^{-} \right) - ig \left( I_{3} C_{W} - \f{Y}{2} \f{S_{W}^{2}}{C_{W}}\right) - ig S_{W} \left( I_{3} + \f{Y}{2}  \right) A_{\mu}
\end{align*}
Theo hệ thức Gell-Mann-Nishijima
\begin{align*}
	Q = I_{3} + \f{Y}{2}  \Rightarrow \f{Y}{2} = Q - I_{3}
\end{align*}
Ta có 
\begin{align*}
	D_{\mu} 
	&= \partial_{\mu} - \f{ig}{\sqrt{2}} \left( T^{+} W_{\mu}^{+} + T^{-} W_{\mu}^{-} \right) - \f{ig}{C_{W}} \left[ I_{3} C_{W}^{2} - \left( Q - I_{3} \right) S_{W}^{2} \right] Z_{\mu} - ig S_{W} \left( I_{3} - Q + I_{3} \right) A_{\mu}\\
	&= \partial_{\mu} - \f{ig}{\sqrt{2}} \left( T^{+} W_{\mu}^{+} + T^{-} W_{\mu}^{-} \right) - \f{ig}{C_{W}}\left( I_{3} - QS_{W}^{2} \right)Z_{\mu} - igS_{W}Q A_{\mu}
\end{align*}
ta đặt $e = g S_{W} = g' C_{W}$.
Nên ta có
\begin{align*}
	D_{\mu} = \partial_{\mu} - \f{ig}{\sqrt{2}} \left( T^{+} W_{\mu}^{+} + T^{-} W_{\mu}^{-} \right) - \f{ig}{C_{W}} \left( I_{3} - Q S_{W}^{2} \right) Z_{\mu} - ie Q A_{\mu}
\end{align*}


\subsection*{Viết Lagrange cho quá trình phân rã sau}
\begin{align*}
	W^{\pm} \rightarrow \tau \overline{\nu_{\tau}}
\end{align*}

Ta có Lagrange cho mô hình chuẩn tổng quát miêu tả toàn bộ quá trình
\begin{align*}
	\mathcal{L} = \overline{\psi} \left( i\slashed{\partial} - m \right) \psi - \f{1}{4}F^{\mu\nu}F_{\mu\nu}
\end{align*}
thay $\psi = \Psi$ là các đa tuyến, ta được
\begin{align*}
	\mathcal{L}
	 & = i \overline{L}^{i} \slashed{D} L^{i} + i \overline{Q_{L}}^{i} \slashed{D} Q_{L}^{i} + i \overline{E_{R}}^{i} \slashed{D} E_{R}^{i} + i \overline{U_{L}}^{i} \slashed{D} U_{L}^{i} + i \overline{D_{L}}^{i} \slashed{D} D_{L}^{i} \\
	 & - \f{1}{4}G^{a \, \mu\nu}G_{\mu\nu}^{a} - \f{1}{4}W^{a \, \mu\nu}W_{\mu\nu}^{a} - \f{1}{4}B^{\mu\nu}B_{\mu\nu}
\end{align*}
mà quá trình phân rã $W^{\pm} \rightarrow \tau \overline{\nu_{\tau}}$, thành phần phải có là  $\tau,\nu_{\tau}$. Ta có
\begin{align*}
	\mathcal{L} \supset \mathcal{L}_{W^{\pm} \rightarrow \tau \nu_{\tau}} = i \overline{L}^{3} \slashed{D} L^{3} = i \overline{L}^{3} \gamma^{\mu} D_{\mu} L^{3}, \tag{1}
\end{align*}
trong đó
\begin{align*}
	D_{\mu}(f) \propto \partial_{\mu} - i g \left( T^{+} W_{\mu}^{+} + T^{-} W_{\mu}^{-} \right) \tag{2}
\end{align*}
và $m = 0$.	Thay (2) vào (1), ta được
\begin{align*}
	\mathcal{L}_{W^{\pm} \rightarrow \tau \nu_{\tau}}
	 & \subset i \overline{L}^{3} \gamma^{\mu} \left[\partial_{\mu} - i g \left( T^{+} W_{\mu}^{+} + T^{-} W_{\mu}^{-} \right)\right] L^{3}                                         \\
	 & \subset i \overline{L}^{3} \gamma^{\mu} \partial_{\mu} L^{3} - i \overline{L}^{3} \gamma^{\mu} i g \left( T^{+} W_{\mu}^{+} + T^{-} W_{\mu}^{-} \right) L^{3} \\
	 & \subset - i \overline{L}^{3} \gamma^{\mu}  i g \left( T^{+} W_{\mu}^{+} + T^{-} W_{\mu}^{-} \right) L^{3}                                                      \\
	 & \subset - i
	\begin{pNiceMatrix}
		\overline{\tau} & \overline{\nu_{\tau}}
	\end{pNiceMatrix}
	\gamma^{\mu}ig \left( T^{+} W_{\mu}^{+} + T^{-} W_{\mu}^{-} \right)
	\begin{pNiceMatrix}
		\tau \\
		\nu_{\tau}
	\end{pNiceMatrix}
\end{align*}
Xét $\left( T^{+} W_{\mu}^{+} + T^{-} W_{\mu}^{-} \right)$
\begin{align*}
	\left( T^{+} W_{\mu}^{+} + T^{-} W_{\mu}^{-} \right)
	 & =
	\begin{pNiceMatrix}
		0           & W_{\mu}^{+} \\
		W_{\mu}^{-} & 0
	\end{pNiceMatrix}
\end{align*}
Ta có
\begin{align*}
	\mathcal{L}_{W^{\pm} \rightarrow \tau \nu_{\tau}}
	 & \subset  g (\overline{\tau} W_{\mu}^{+}\nu_{\tau}  + \overline{\nu_{\tau}}W_{\mu}^{-}\tau ).
\end{align*}
\subsection*{Tính số hạng khối lượng $M_{h}$}
Ta có Lagrange cho trường Higgs
\begin{align*}
	\mathcal{L}_{\text{Higgs}} = \left( D^{\mu}\phi \right)^{\dagger} \left( D_{\mu}\phi \right) - V(\phi), \tag{1}
\end{align*}
trong đó
\begin{align*}
	\phi = \f{1}{\sqrt{2}}
	\begin{pNiceMatrix}
		0 \\
		v + h(x)
	\end{pNiceMatrix}
\end{align*}
và
\begin{align*}
	V              & = -\mu^{2} \abs{\phi}^{2} + \lambda \abs{\phi}^{4}, \\
	\abs{\phi}^{2} & = \phi^{\dagger} \phi.
\end{align*}
Khai triển thế $V$ ta được
\begin{align*}
	V
	 & = -\mu^{2} \phi^{\dagger} \phi + \lambda \left(\phi^{\dagger} \phi\right)^{2}                                                                                      \\
	 & = -\f{\mu^{2}}{2}
	\begin{pNiceMatrix}
		0 & v + h(x)
	\end{pNiceMatrix}
	\begin{pNiceMatrix}
		0 \\
		v + h(x)
	\end{pNiceMatrix}
	+ \lambda\Biggl[
		\begin{pNiceMatrix}
			0 & v + h(x)
		\end{pNiceMatrix}
		\begin{pNiceMatrix}
			0 \\
			v + h(x)
		\end{pNiceMatrix}
	\Biggr]^{2}                                                                                                                                                           \\
	 & = -\f{\mu^{2}}{2} \left(v^{2} + h^{2}(x) + 2vh(x) \right) + \lambda \left(v^{2} + h^{2}(x) + 2vh(x)\right)^{2}                                                     \\
	 & = -\f{\mu^{2}}{2} \left(v^{2} + h^{2}(x) + 2vh(x)\right) + \lambda \left( v^{4} + h^{4}(x) + 4 v^{2}h^{2}(x) + 2 v^{2}h^{2}(x) + 2 v^{3}h(x) + 4 vh^{3}(x) \right)
\end{align*}
Khối lượng $M_{h}$ là hệ số đứng trước số hạng $\f{1}{2}h^{2}(x)$, ta có
\begin{align*}
	M_{h}^{2} 
	&= -2\mu^{2} + 6\lambda v^{2} \\
	&= -2\mu^{2} + 6\lambda \f{u^{2}}{\lambda}
\end{align*}
nên ta có khối lượng của Higgs là
\begin{align*}
	M_{\text{Higgs}} = \mu\sqrt{2}
\end{align*}

\subsection*{Viết Lagrange tương tác cho $hZ_{\mu}Z^{\mu}, hhZ_{\mu}Z^{\mu}$ }

Xét
\begin{align*}
	\mathcal{L_{\text{Higgs}}} \supset \left( D^{\mu}\phi \right)^{\dagger} \left( D_{\mu}\phi \right)
\end{align*}
trong đó
\begin{align*}
	D_{\mu}(f)
	 & \supset - \f{ig}{C_{W}}I_{3} Z_{\mu}
\end{align*}
Ta có
\begin{align*}
	\mathcal{L_{\text{Higgs}}}
	 & \supset \left(- \f{ig}{C_{W}} \right)^{2}
	\Bigg[
	I_{3} Z_{\mu}
	\begin{pNiceMatrix}
		0 \\
		\frac{1}{\sqrt{2}}(v + h(x))
	\end{pNiceMatrix}^{\dagger}
	I_{3} Z^{\mu}
	\begin{pNiceMatrix}
		0 \\
		\frac{1}{\sqrt{2}}(v + h(x))
	\end{pNiceMatrix}
	\Bigg]                                                                                                                         \\
	 & \supset
	\left(- \f{ig}{C_{W}} \right)^{2}
	\Bigg[
		\begin{pNiceMatrix}
			0 & -\frac{1}{{2\sqrt{2}}}(v + h(x))
		\end{pNiceMatrix}
		\begin{pNiceMatrix}
			0 \\
			-\frac{1}{{2\sqrt{2}}}(v + h(x))
		\end{pNiceMatrix}
		Z_{\mu}
		Z^{\mu}
	\Bigg]                                                                                                                         \\
	 & \supset \left(- \f{ig}{C_{W}} \right)^{2} \left[ \f{1}{8} \left( v^{2} + 2vh(x) + h^{2}(x) \right) \right] Z_{\mu}Z^{\mu}   \\
	 & \supset \f{1}{8}\left(- \f{ig}{C_{W}} \right)^{2} \left( v^{2}Z_{\mu}Z^{\mu} + 2vhZ_{\mu}Z^{\mu} + hhZ_{\mu}Z^{\mu} \right)
\end{align*}



















\end{document}
